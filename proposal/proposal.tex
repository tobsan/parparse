\documentclass[a4paper,12pt]{article}
\usepackage[utf8]{inputenc}

\title{Parallel parsing in Yi \\ 
       \small{Project proposal for master's thesis}\thanks{Aim for 
              specialization in Algorithms, Languages and Logic}}
\author{Tobias Olausson \\ \small{gusolaut@student.gu.se}}
\date{}

\begin{document}
\maketitle

\section*{Background}
Syntax highlighting is a key feature in modern day text editors. Programmers
want to get information about how well their code conforms to the programming
language grammar, but without having to compile or run their code. Most text
editors use regular expressions to find keywords and highlight other key aspects
of the code. While this is fast, it cannot provide any information beyond the
lexical syntax of the language.

A nicer approach would be to use the language's own grammar to check against the
written code. If the editor had a lexer and parser integrated for the language,
it could not only markup the relevant parts of the code, it would also allow the
programmer to use any metadata that the parser generates to manipulate the code
on an abstract level.

\section*{Problem description}
Recent research has shown that parsing of programming language text can be done
efficiently by using a divide-and-conquer algorithm\cite{parparsepaper}. Another
MSc thesis by Hugo and Hansson has created a way to generate an incremental
divide-and-conquer lexer given a context-free grammar\cite{hugohansson}.

The aim of this project is to combine these two results and implement them into
the Yi text editor\cite{yieditor}. The result would be an editor that, in an
efficient manner, could give meaningful syntax information given any
context-free grammar.

\subsection*{Delimitations}
The project will be concerned with only the lexer generator and parsing, other
parts of the Yi editor will be left untouched as much as possible. Cool
applications using the syntax information generated by the lexer and parser will
not be a priority, but rather a bonus if time allows for it.

\subsection*{Methodology}
This project is mainly concerned with implementing theories already proven into
a practical application. Obviously, before implementing anything the existing Yi
code base must be carefully examined. When it comes to the actual code, much
work will revolve around testing and debugging the code. The resulting parser
should be easily checked against existing grammars both in the editor and in a
more stand-alone environment.

\subsection*{Project plan}
The project is planned for the spring semester of 2014.  Implementation and
integration of the components will be performed in several steps. The lexer
generator obviously has to be in place before the parser can be implemented.
Very coarse steps:
\begin{enumerate}
    \item{Study the background papers and Yi source code}
    \item{Integrate the lexer generator into Yi}
    \item{Implement the parallel parser into Yi}
    \item{Use syntactic information for cool stuff (if time permits it)}
\end{enumerate}

\subsection*{Supervisor}
This project was suggested by Jean-Philippe Bernardy, and as one of the authors
of the recent paper on parallel parsing and as one of the main contributors to
Yi, he is likely the most suitable for supervising the project.

\section*{Prerequisities}
This will be a project in computing science at University of Gothenburg. Since
the Yi editor and most other software, such as BNFC\cite{bnfc}, involved in this
project will be written in Haskell, good knowledge in Haskell is required. Since
it also deals with the initial tasks of a compiler, knowledge in programming
languages and compilers is also required. Courses that fulfil these requirements
include advanced functional programming (DIT260), programming languages
(DIT229/230) and compiler construction (DIT300). I have read all of these
courses.

\begin{thebibliography}{9}

\bibitem{parparsepaper}
    Efficient Divide-and-Conquer Parsing of Practical Context-Free Languages \\
    Jean-Philippe Bernardy, Koen Claessen \\
    ICFP 2013

\bibitem{hugohansson}
    A generator of incremental divide-and-conquer lexers \\
    Jonas Hugo, Christoffer Hansson \\
    MSc Thesis, Chalmers University of Technology, draft as of 2013-11-21

\bibitem{yieditor}
    Yi: an editor in Haskell for Haskell. \\
    Jean-Philippe Bernardy \\
    In Proc. of the first ACM SIGPLAN symposium on Haskell, pages 61–62. ACM,
    2008.

\bibitem{bnfc}
    The BNF Converter \\
    http://bnfc.digitalgrammars.com/

\end{thebibliography}

\end{document}
