\documentclass[a4paper,12pt]{article}
\usepackage[utf8]{inputenc}

\title{Incremental and parallel parsing in Yi \\ 
       \small{Planning report}}
\author{Tobias Olausson \\ \small{gusolaut@student.gu.se}}
\date{}

\begin{document}
\maketitle

\section*{Preliminary title}
The preliminary title for this thesis is \textit{Incremental and parallel
parsing in Yi}. An alternative, in case time does not permit implementation into
Yi would simply be \textit{Implementing incremental and parallel parsing}.

\section*{Background}
Syntax highlighting is a key feature in modern day text editors. Programmers
want to get information about how well their code conforms to the programming
language grammar, but without having to compile or run their code. Instead of
using the widely adopted regular expression approach to syntax highlighting, we
could use a lexer and parser that is both incremental and fast to give us syntax
information in the form of an abstract syntax tree. Such a tree would provide
more information about the given code, and makes for the ability to manipulate
the code in new, cool ways.

\section*{Aim for the work}
The aim for this thesis project is to write an incremental, parallelizable
parser in Haskell, which can be connected to the lexer written by Hugo and
Hansson\cite{hugohansson}. The ultimate goal would be to put these both to the
test by integrating them into the Yi text editor.

\section*{Assignment}
Recent research by Claessen and Bernardy\cite{parparsepaper} has shown that
parsing of programming language text adhering to a context-free grammar can be
done efficiently by using a divide-and-conquer algorithm, improved from
Valiant's \cite{valiant}.  Another MSc thesis by Hugo and Hansson has given an
incremental divide-and-conquer lexer for a context-free grammar. The lexer works
in an incremental manner, but the proof-of-concept implementation of the parser
is not. The main task is to reimplement the parser in an incremental fashion,
and to integrate the lexer and parser to work together. 

\section*{Delimitations}
The project will be concerned with only the lexer generator and parsing, other
parts of the Yi editor will be left untouched as much as possible. Cool
applications using the syntax information generated by the lexer and parser will
not be a priority, but rather a bonus if time allows for it.

\section*{Methodology}
This project is mainly concerned with implementing theories already proven into
a practical application. Obviously, before implementing anything the existing Yi
code base must be carefully examined. When it comes to the actual code, much
work will revolve around testing and debugging the code, in an iterative manner.
The resulting parser should be easily checked against existing grammars both in
the editor and in a more stand-alone environment. 

\section*{Time plan}
The project is planned for the spring semester of 2014. Implementation and
integration of the components will be performed in several steps. The time plan
will be continuosly checked and possibly revised during the work, together with
the supervisor.

\subsection*{Background data}
The first 1-2 weeks will be spent studying background papers and getting
familiar with the lexer written by Hugo and Hansson. 

\subsection*{Implement incremental parser}
The incremental parser should be implemented and tested by March 9th. The parser
will utilize finger trees as its primary data structure, just as the lexer does.
There are two major steps in implementing the parser.

First, the existing proof-of-concept implementation of parsing using matrix
multiplication has to be extended in order to be usable from an incremental
parsing approach where merging two sub-trees should be possible. This means
implementing some new functionality into BNFC\cite{bnfc}.

Second, a parsing monoid that can be used in the finger tree as well as with the
matrix multiplication code in BNFC has to be written. The monoid should
transform lexemes into the abstract syntax tree that the parser will output as a
result, and will use the matrix multiplication to do this.

\subsection*{Generalize the lexer generator}
The currently working code from the lexer generator by Hugo and Hansson is
specific to the Java programming language and uses a state machine generated by
Alex internally. This can be generalized, since the state machine and data types
for tokens are the only language-specific parts of their code. It should not be
too much work to turn their results into something that BNFC can generate for
any given language. Of course, a reasonable code generation from BNFC will also
include the aforementioned parser. An estimated time frame for this is two
weeks.

\subsection*{Integration with Yi}
Once the lexer and parser works, and can be generated given any context-free
grammar, this should be integrated into the Yi text editor\cite{yieditor}, where
incremental parsing could be used to check syntax correctness without having to
call the compiler for the language. If time permits, this integration can be
extended with functions that use the syntactic information to do cool stuff,
such as AST transformations etc. 

\subsection*{Report}
By the middle of April, most of the coding should be finished, and focus should
rather be on writing the report. However, we expect some debugging and
beautifying of the code to take place during the writing phase, as documenting a
piece of work usually reveals minor shortcomings.

\begin{thebibliography}{9}

\bibitem{parparsepaper}
    Efficient Divide-and-Conquer Parsing of Practical Context-Free Languages \\
    Jean-Philippe Bernardy, Koen Claessen \\
    ICFP 2013

\bibitem{hugohansson}
    A generator of incremental divide-and-conquer lexers \\
    Jonas Hugo, Christoffer Hansson \\
    MSc Thesis, Chalmers University of Technology, draft as of February 2014

\bibitem{valiant}
    General Context-Free Recognition in Less than Cubic Time \\
    Leslie G. Valiant \\ 
    Journal of Computer and System Sciences 10, 1975

\bibitem{yieditor}
    Yi: an editor in Haskell for Haskell. \\
    Jean-Philippe Bernardy \\
    In Proc. of the first ACM SIGPLAN symposium on Haskell, pages 61–62. ACM,
    2008.

\bibitem{bnfc}
    The BNF Converter \\
    http://bnfc.digitalgrammars.com/

\end{thebibliography}

\end{document}

