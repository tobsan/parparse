\documentclass[a4paper,12pt]{article}
\usepackage[utf8]{inputenc}

\title{Incremental and parallel parsing in Yi \\ 
       \small{Planning report}}
\author{Tobias Olausson \\ \small{gusolaut@student.gu.se}}
\date{}

\begin{document}
\maketitle
Note: references are omitted here; see the project proposal for details about
those. Some material in this report has been reused from the proposal.

\section*{Preliminary title}
The preliminary title for this thesis is \textit{Incremental and parallel
parsing in Yi}. An alternative, in case time doesn't permit implementation into
Yi would simply be \textit{Implementing incremental and parallel parsing}.

\section*{Background}
Syntax highlighting is a key feature in modern day text editors. Programmers
want to get information about how well their code conforms to the programming
language grammar, but without having to compile or run their code. Instead of
using the widely adopted regular expression approach to syntax highlighting, we
could use a fast lexer and parser to give us syntax information in the form of
an abstract syntax tree. Such a tree would provide more information about the
given code, and makes for the ability to manipulate the code in new, cool ways.

\section*{Aim for the work}
The aim for this thesis project is to write an incremental, parallelizable
parser in Haskell, which can be connected to the lexer written by Hugo and
Hansson. The ultimate goal would be to put these both to the test by integrating
them into the Yi text editor.

\section*{The assignment}
Recent research by Claessen and Bernardy has shown that parsing of programming
language text can be done efficiently by using a divide-and-conquer algorithm.
Another MSc thesis by Hugo and Hansson has given an incremental
divide-and-conquer lexer for a context-free grammar. The lexer is written in
an incremental manner, but the proof-of-concept implementation of the parser is
not. The main task is to reimplement the parser in an incremental fashion, and
to integrate the lexer and parser to work as one. 

\section*{Delimitations}
The project will be concerned with only the lexer generator and parsing, other
parts of the Yi editor will be left untouched as much as possible. Cool
applications using the syntax information generated by the lexer and parser will
not be a priority, but rather a bonus if time allows for it.

\section*{Methodology}
% REALLY?
This project is mainly concerned with implementing theories already proven into
a practical application. Obviously, before implementing anything the existing Yi
code base must be carefully examined. When it comes to the actual code, much
work will revolve around testing and debugging the code. The resulting parser
should be easily checked against existing grammars both in the editor and in a
more stand-alone environment.

\section*{Time plan}
The project is planned for the spring semester of 2014. Implementation and
integration of the components will be performed in several steps. Time plan will
be revised during the work, together with the supervisor.
\begin{enumerate}
    \item{Study the background papers (first 1-2 weeks)}
    \item{Implement the incremental parallel parser (should be done by end of
          february}
        \begin{itemize}
            \item{Extend the matrix computation code}
            \item{Write a parsing monoid that can be used together with BNFC
                  generated code}
            \item{Combine the lexer and parser}
        \end{itemize}
    \item{Generalize (templatify) the lexer generator (should take no more than
          two weeks)}
    \item{Integrate the parser (with the lexer) into Yi (until the middle of
          april)}
    \item{Use syntactic information for cool stuff (if time permits it)}
\end{enumerate}
By the middle of April, not so much code should be done, but focus should rather
be on writing the report. Some coding will be done, however, since one usually
finds some mistakes when writing about the code.

\end{document}
