\documentclass[a4paper,12pt,twosided]{report}
\usepackage[utf8]{inputenc}
\usepackage{datetime}
\usepackage[pdftex]{graphicx}
\usepackage{epstopdf}

\newdateformat{monthdate}{\monthname[\THEMONTH] \THEYEAR}

\begin{document}

% Adapted the GU template from word to LaTeX
\begin{titlepage}
\thispagestyle{empty}

\begin{center}
\includegraphics[width=\textwidth]{gulogo.pdf}
\end{center}

\vfill

\begin{flushleft}
{\LARGE Implementing incremental and parallel parsing} \\[0.2cm]
{\large \textit{Master of Science Thesis in Computer Science}}\\[3cm]

{\Huge \textsc{Tobias Olausson}}

\vfill

University of Gothenburg \\
Chalmers University of Technology \\
Department of Computer Science and Engineering \\
Göteborg, Sweden, \monthdate\today
\end{flushleft}

\newpage
\thispagestyle{plain}

\noindent The Author grants to Chalmers University of Technology and University
of Gothenburg  the non-exclusive right to publish the Work electronically and in
a non-commercial purpose make it accessible on the Internet.  The Author
warrants that he is the author to the Work, and warrants that the Work does
not contain text, pictures or other material that violates copyright law.\\

\noindent The Author shall, when transferring the rights of the Work to a third
party (for example a publisher or a company), acknowledge the third party about
this agreement. If the Author has signed a copyright agreement with a third
party regarding the Work, the Author warrants hereby that he has obtained
any necessary permission from this third party to let Chalmers University of
Technology and University of Gothenburg  store the Work electronically and make
it accessible on the Internet. \\[2cm]

\noindent \textbf{Implementing incremental and parallel parsing}\\
\noindent \textsc{Tobias Olausson} \\

\noindent \copyright \textsc{ Tobias Olausson}, \monthdate\today \\

\noindent Examiner: \textsc{Patrik Jansson} \\

\noindent University of Gothenburg \\
Chalmers University of Technology \\
Department of Computer Science and Engineering \\
SE-412 96 Göteborg \\
Sweden \\
Telephone + 46 (0)31-772 1000 \\

\vfill

\noindent Department of Computer Science and Engineering \\
Göteborg, Sweden, \monthdate\today
\end{titlepage}



\begin{abstract}
This is an abstract
\end{abstract}

\tableofcontents

%
% NEW CHAPTER
%

\chapter{Background}

\section{Introduction}
The topic of this thesis is to do \textbf{parsing} in an \textbf{incremental}
fashion that can easily be parallelizable. To parse is a to check if some given
input corresponds to a certain language's grammar, and in this case context-free
programming languages' grammars.

% What is the topic? Describe each word in the title.
% Something like a motivation on why this is interesting at all.
\subsection{Incrementality}
Doing something incrementally means that one does it step by step, and not
longer than neccessary.

\subsection{Parallelism}
\subsection{Parsing}
\subsection{Motivation}
In compilers, lexing and parsing are the two first phases. The output of these
is an abstract syntax tree (AST) which is fed to the next phase of the compiler.
But an AST could also provide useful feedback for programmers, already in their
editor, if the code could be lexed and parsed fast enough. With a lexer and
parser that is incremental and that can also be parallelized could real-time
feedback in the form of an AST easily be provided to the programmer. Most
current text editors give syntax feedback based on regular expressions, which
does not yield any information about depth or the surrounding AST.

\section{Lexing}
Shortly describe LexGen and its relevance.

\section{Context-free grammars}
\subsection{Chomsky Normal Form}

\section{Parsing}
\subsection{CYK algorithm}
\subsection{Improvement by Bernardy \& Claessen}

\section{Dependently typed programming}
What is dependently typed programming, and how can it be used in Haskell.
\subsection{Kinds, Types and Values}

%
% NEW CHAPTER
%

\chapter{Implementation}

\section{Finger trees}
\section{Measuring and Monoids}
\subsection{Pipeline of measures}
An illustration would be good here

\section{Lexing}
\subsection{Position information}

\section{Parsing}
\subsection{BNFC}
\subsection{Dependently typed programming with charts}
\subsection{Oracle and unsafePerformIO}


\section{Final product}
\subsection{Testing}

%
% NEW CHAPTER
%

\chapter{Results}

\section{Measurements}
How fast is it? What is the complexity?

%
% NEW CHAPTER
%
\chapter{Discussion}

% TODO: Describe this in some other way. 
\section{Implementation}
% Describe sort of chronologically?
\subsection{Too many result branches}
\subsection{LexGen -- Alex discrepancy}
\subsection{Position information}

\section{Improvements}
\section{Future work}

%
% BIBLIOGRAPHY
%
\bibliographystyle{plain}
\bibliography{bibliography.bib}

%
% ANY APPENDICES HERE
%

\end{document}
